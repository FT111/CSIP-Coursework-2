
\documentclass{article}
\renewcommand{\thesubsection}{\thesection.\alph{subsection}}
\usepackage[english]{babel}
\usepackage{amssymb}
\usepackage{steinmetz}
\usepackage[a4paper,top=1cm,bottom=2cm,left=2cm,right=2cm,marginparwidth=1.75cm]{geometry}

\usepackage{amsmath}
\usepackage{graphicx}
\usepackage[colorlinks=true, allcolors=blue]{hyperref}
\usepackage{tikz}
\usepackage{pgfplots}
\graphicspath{{images/}}
\DeclareGraphicsExtensions{.pdf,.jpeg,.png}

\title{CS1IP Coursework 2 \\ \small{Sort Comparison}}
\author{Freddie Taylor}
\date{}
\begin{document}
\maketitle

    \includegraphics[width=\linewidth]{ComparisonChart}
\vspace{5mm}
\newline

\noindent The two sorts initially perform similarly - with the 10 and 100 item benchmarks lacking
any easily perceivable speed differences, despite the different time complexities.
This is because the O($n \log_2 n$) complexity of the merge sort and O($n^2$) of bubble
result in an extremely similar number of worst-case comparison operations at lower item counts
    \begin{center}
        \begin{tabular}{ c c c c}
            & 10 items & 100 items & 10000 items \\
            Merge Sort & $\approx 66$ comparisons & $\approx 664$ comparisons & $\approx 132{\small,}877$ \\
            Bubble Sort & $\approx 100$ comparisons & $\approx 1000$ comparisons & $\approx 100{\small,}000{\small,}000$
        \end{tabular}
    \end{center}
\noindent The differences in comparison count where items $\leq100$ between the two sorts shown above are
    relatively inconsequential to modern CPUs unless performed many times, as shown by the results
    all rounding approximately to a millisecond.
    This suggests it's not always that beneficial to use a more complex, `faster' sort when you are
    handling smaller arrays due to the marginal improvements found. \\

    \noindent The major performance difference is observed when the number of items increases to $10{\small,}000$.
    Due to merge sort's logarithmic time complexity, it performs, at most, 99,867,123 less comparison
    operations, showing that the performance difference of the two significantly increases as item count
    increases.
\end{document}
